% ==== Appendix A ==== %
\section[Appendix]{Appendix}
\label{sec:app_a}

\subsection{Source Code and Reproducibility}
\textbf{GitHub Repository:} \href{https://github.com/jeevandhakal/5580-dm-a4-regression-ts}{Electricity Demand Regression Repository}

\subsection{Experimental Hardware Configuration}
\begin{table}[H]
	\centering
	\caption{Experimental Hardware Configuration}
	\label{tab:experimental_setup}
	\begin{tabular}{lp{0.7\textwidth}}
		\toprule
		\textbf{Category} & \textbf{Configuration} \\
		\midrule
		CPU & Intel Core i9-13900K processor with 24 cores (8P+16E), boost frequency up to 5.8 GHz. \\
		RAM & 32 GB DDR5 memory operating at 4800 MT/s. \\
		GPU & NVIDIA GeForce RTX 4070 with 8 GB GDDR6X VRAM, 5888 CUDA cores. \\
		OS & Zorin OS Linux distribution (Kernel 6.x) with CUDA 12.5.1 and cuDNN 9 support. \\
		\bottomrule
	\end{tabular}
\end{table}

\subsection{Exploratory Data Analysis (EDA)}

\begin{figure}[H]
    \centering
    \includegraphics[width=0.9\linewidth]{figures/target_time_series.png}
    \caption{Full historical electricity consumption time series utilized in this study.}
    \label{fig:app_target_ts}
\end{figure}

\begin{figure}[H]
    \centering
    \includegraphics[width=0.9\linewidth]{figures/target_distribution.png}
    \caption{Probability density function of the target variable (Consumption), showing a slight right-skew.}
    \label{fig:app_target_dist}
\end{figure}

\begin{figure}[H]
    \centering
    \includegraphics[width=0.9\linewidth]{figures/correlation_heatmap.png}
    \caption{Correlation matrix of lags and cyclical features against the target variable.}
    \label{fig:app_corr_heatmap}
\end{figure}

\subsection{Time Series Decomposition}

\begin{figure}[H]
    \centering
    \includegraphics[width=0.9\linewidth]{figures/ts_decomposition.png}
    \caption{Seasonal decomposition of the dataset into Trend, Seasonal (Daily), and Random Noise.}
    \label{fig:app_ts_decomp_full}
\end{figure}

\begin{figure}[H]
    \centering
    \includegraphics[width=0.9\linewidth]{figures/acf_pacf.png}
    \caption{Autocorrelation (ACF) and Partial Autocorrelation (PACF) plots used for feature lag selection.}
    \label{fig:app_acf_pacf}
\end{figure}

\subsection{Optuna Hyperparameter Optimization Logs}

\begin{figure}[H]
    \centering
    \includegraphics[width=0.9\linewidth]{figures/tuning_results_2.png}
    \caption{Optuna objective expansion showing the convergence toward minimal MAPE across 100 trials.}
    \label{fig:app_optuna_convergence}
\end{figure}

\begin{figure}[H]
    \centering
    \includegraphics[width=0.9\linewidth]{figures/optimization_history.png}
    \caption{Progress of the Optuna study across 100 trials, minimizing the MAPE objective.}
    \label{fig:optuna_history}
\end{figure}

\begin{figure}[H]
    \centering
    \includegraphics[width=0.9\linewidth]{figures/slice_plot.png}
    \caption{Parameter slice plot showing the impact of specific hyperparameter ranges on the objective value.}
    \label{fig:optuna_slice_full}
\end{figure}

\subsection{Final Model Benchmarks and Sample Predictions}

\begin{figure}[H]
    \centering
    \includegraphics[width=0.9\linewidth]{figures/advanced_model_comparison.png}
    \caption{Full comparison of all nine models across MAPE, MAE, and RMSE metrics.}
    \label{fig:app_full_comparison}
\end{figure}

\begin{figure}[H]
    \centering
    \includegraphics[width=0.9\linewidth]{figures/liear_regression_prediction_first_200_test.png}
    \caption{Visual calibration showing True vs. Predicted values for the Linear Regression baseline on the test set.}
    \label{fig:app_lr_pred}
\end{figure}
