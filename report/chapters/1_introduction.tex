\section{Mandatory Assignment Answers}
\label{sec:mandatory_answers}

\subsection{Supervised Learning and Business Applications}
\textbf{Q1: Supervised Learning Definition}\\
Supervised learning is a data analysis paradigm used to extract models that describe important data classes or forecast future trends. It relies on a labeled dataset where the algorithm learns from known pairs of inputs and outputs. 
\begin{itemize}
    \item \textbf{Classification:} Predicts categorical class labels (e.g., "safe" or "risky").
    \item \textbf{Prediction (Regression):} Forecasts continuous-valued functions or numeric values.
\end{itemize}

\textbf{Q2: Italian Clothing Company Case}\\
In the business context of product demand, a clothing brand used linear regression to quantify the relationship between advertising spend and sales. The resulting model was:
\begin{equation}
    Sales = 168 + 23 \times (Advertising)
\end{equation}

\subsection{Prediction Techniques and Time-Series Analysis}
\textbf{Q3: Popular Prediction Techniques}\\
\begin{itemize}
    \item \textbf{Linear Regression:} Minimizes error in $Y = \beta_0 + \beta_1x$.
    \item \textbf{Neural Networks:} Uses \textbf{Forward Propagation} for activations and \textbf{Back Propagation} to adjust weights based on errors.
    \item \textbf{Support Vector Machine (SVM):} Creates a hyperplane to separate data in high-dimensional space.
    \item \textbf{Random Forests:} An ensemble of decision trees that improves robustness by averaging multiple results.
\end{itemize}

\textbf{Q4: Time-Series Methodology}\\
Time-series analysis focuses on predicting future values based solely on previously observed values. \textbf{Decomposition} splits data into:
\begin{itemize}
    \item \textbf{Seasonal:} Repeating patterns within a fixed period.
    \item \textbf{Trend:} Long-term underlying direction.
    \item \textbf{Noise:} Random residuals after removing trend and seasonality.
\end{itemize}

\textbf{Q5: Popular Time-Series Prediction Techniques:}\\
\begin{itemize}
    \item \textbf{Holt-Winters:} Exponential smoothing for data with trend and seasonality.
    \item \textbf{ARIMA/SARIMA:} Autoregressive integrated moving average (seasonal variant included).
    \item \textbf{Exponential Smoothing (ETS):} General class of smoothing models.
    \item \textbf{Prophet:} Developed by Facebook, handles seasonality, holidays, and trends.
    \item \textbf{LSTM (Deep Learning):} Recurrent neural network for sequence prediction.
    \item \textbf{TBATS:} Handles multiple seasonalities and complex patterns.
    \item \textbf{Vector Autoregression (VAR):} For multivariate time series.
\end{itemize}

\subsection[Performance Metrics]{Performance Metrics and Historical Benchmarks}
\textbf{Q6: Time-Series Performance Metrics:}\\
\begin{itemize}
    \item \textbf{Mean Absolute Percentage Error (MAPE):} Percentage-based error, easy to interpret.
    \item \textbf{Root Mean Squared Error (RMSE):} Penalizes larger errors more heavily.
    \item \textbf{Mean Absolute Error (MAE):} Average absolute error, robust to outliers.
    \item \textbf{Mean Absolute Scaled Error (MASE):} Scale-independent, good for comparing across series.
    \item \textbf{Symmetric MAPE (sMAPE):} Modified version to address MAPE’s asymmetry.
    \item \textbf{R-squared ($R^2$):} Proportion of variance explained by the model.
\end{itemize}

\subsection[Beer Dataset]{The Australian Beer Dataset Case Study (Q7)}
Analysis of monthly Australian beer consumption (1956--1995) demonstrates how \textbf{Time Series Decomposition} converts complex fluctuations into predictable patterns. By isolating the \textbf{Trend} (long-term growth), \textbf{Seasonal} (repeating annual peaks), and \textbf{Noise} components, demand forecasting becomes statistically grounded.

Modeling benchmarks showed that \textbf{Holt-Winters} effectively captured these cycles, yielding a highly accurate \textbf{MAPE of 5.44\%}, while the \textbf{ARIMA(4, 0, 0)} model leveraged four lags for its autoregressive estimations. Ultimately, this analysis proves that brewery consumption is cyclically structured rather than random, enabling optimized 12-month inventory and production planning.

However, the evaluation process lacks rigor. Accuracy metrics are reported without clear definitions or context, making interpretation difficult. There is no mention of a \textbf{train-test split} or out-of-sample validation, which risks overfitting and reduces confidence in the model's real-world performance. Additionally, model selection---particularly for ARIMA parameters---appears arbitrary without support from autocorrelation plots or automated selection tools.

To improve, we recommend implementing a formal validation framework using a \textbf{holdout set} and consistent error metrics such as \textbf{MAPE} or \textbf{RMSE}. Using \texttt{auto.arima()} for parameter optimization and performing residual diagnostics would strengthen model reliability. Exploring Seasonal ARIMA (\textbf{SARIMA}) could better capture the data's strong monthly patterns.

\subsection[Model Experiments]{Custom Comparative Findings: Electricity Demand (Q8)}
Our experimental results on high-frequency electricity consumption data revealed a significant performance gap between classical and modern ensemble methods. Based on our final evaluations:
\begin{itemize}
	\item \textbf{Deep Learning vs. Kernels:} The \textbf{NN (3-Layer)} achieved a highly competitive \textbf{10.76\% MAPE}, proving that multi-layer connections can effective map complex load surges. However, it required significant computational overhead (~39s).
	\item \textbf{Classical Baseline:} Linear Regression, despite its speed, lagged behind with a \textbf{11.49\% MAPE}, failing to capture the non-linear intraday peaks.
	\item \textbf{Modern Ensemble Performance:} \textbf{LightGBM} emerged as the definitive benchmark leader with a \textbf{9.88\% MAPE}. This demonstrates that for high-resolution datasets, gradient boosting on discretized histograms outperforms both classical ARIMA (which struggled with a 63.36\% error) and deep neural architectures.
\end{itemize}
\textbf{Conclusion:} The choice of architecture is the primary driver of accuracy. While classical models are efficient, they lack the "contextual depth" required to handle the irregular spikes inherent in real-world numeric datasets.